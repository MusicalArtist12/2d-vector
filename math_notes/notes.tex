\documentclass[12pt]{article}
\usepackage[margin=0.5in]{geometry} % essentially #include with options
\usepackage{lmodern}
\usepackage{listings}
\usepackage{amsmath}

\begin{document}

\section{Collision Model}
    \begin{flushleft}
        The impulse over a single $dt$, which for this simulator is some fraction of the individual frame time is as follows. 
    \end{flushleft}
    \[ 
        \vec{\mathbf{J}}_A=
        \int_{t_{0}}^{t_{1}} \vec{\mathbf{F}}_{A}\;dt=
        \Delta p_{a}=\left| \vec{\mathbf{p}}_{B} \right|\cdot 
        \begin{bmatrix} 
            \cos\left( \theta_{p,b} \right) \\ 
            \sin\left( \theta_{p,b} \right) 
        \end{bmatrix} 
        - \left| \vec{\mathbf{p}}_{A} \right|\cdot 
        \begin{bmatrix} 
            \cos\left( \theta_{p,a} \right) \\ 
            \sin\left( \theta_{p,a} \right) 
        \end{bmatrix} 
    \]
    \begin{flushleft}
        The following equations are used to find the angles used above.
    \end{flushleft}
    \[ 
        \theta=\left| \tan^{-1}\left( \frac{A_{y}-B_{y}}{A_{x}-B_{x}} \right) \right|
    \]

    \[
        \theta_{p,b}=\tan^{-1}\left( \frac{p_{b,y}}{p_{b,x}} \right) - \theta
    \]

    \[
        \theta_{p,a}=\tan^{-1}\left( \frac{p_{a,y}}{p_{a,x}} \right) - \theta  
    \]

    \begin{flushleft}
        A note, since the domain of $\tan^{-1}$ is $\left[ -\frac{\pi}{2}, \frac{\pi}{2} \right]$, when the denominator of any of the three previous equations equals zero, the denominator is set to the arbitrarily low value of $0.0001$ to approximate $\lim_{x\to 0}\tan^{-1}\left(\frac{1}{x}\right)$. This is to maintain the sign in the numerator.
    \end{flushleft}

\subsection{Implementation} 

\lstdefinestyle{mystyle}{
    keywordstyle=\textbf\ttfamily,
    basicstyle=\ttfamily\footnotesize,
    breakatwhitespace=false,         
    breaklines=true,                 
    captionpos=b,                    
    keepspaces=true,                                   
    numbersep=5pt,                  
    showspaces=false,                
    showstringspaces=false,
    showtabs=false,                  
    tabsize=2
}

\lstset{style=mystyle}

\begin{lstlisting}[language=C++]
void world::resolveCollision(physObject& objA, physObject& objB, float deltaTime) {
    glm::vec3& forceA = objA.forceVectors.add( objB.getID(), glm::vec3(0.0f) );
    glm::vec3& forceB = objB.forceVectors.add( objA.getID(), glm::vec3(0.0f) );

    float angle = glm::abs(glm::atan((objA.pos.y-objB.pos.y)/(objA.pos.x-objB.pos.x)));

    float momentumAngleA;
    float momentumAngleB;

    if(objA.momentum().x == 0) {
        momentumAngleA = glm::atan(objA.momentum().y/0.0001);
    } else {
        momentumAngleA = glm::atan(objA.momentum().y/objA.momentum().x);
    }

    if(objB.momentum().y == 0) {
        momentumAngleB = glm::atan(objB.momentum().y/0.0001);
    } else {
        momentumAngleB = glm::atan(objB.momentum().y/objB.momentum().x);
    }

    // add other's momentum
    forceA = glm::length(objB.momentum()) * glm::vec3(glm::cos(momentumAngleB - angle), glm::sin(momentumAngleB - angle), 0.0f); 
    forceB = glm::length(objA.momentum()) * glm::vec3(glm::cos(momentumAngleA - angle), glm::sin(momentumAngleA - angle), 0.0f); 

    // remove own momentum
    forceA -= glm::length(objA.momentum()) * glm::vec3(glm::cos(momentumAngleA - angle), glm::sin(momentumAngleA - angle), 0.0f);
    forceB -= glm::length(objB.momentum()) * glm::vec3(glm::cos(momentumAngleB - angle), glm::sin(momentumAngleB - angle), 0.0f);

    forceA /= deltaTime;
    forceB /= deltaTime;
}
    
\end{lstlisting}

\end{document}